\section{Introduction}
Un réseau routier présentant de fortes dégradations impacte inévitablement la
sécurité et le confort des usagers. Cependant, les conséquences d'un mauvais
entretien des routes génère aussi des dépenses supplémentaires pour les
automobilistes, une augmentation de la pollution et nuit à la compétitivité
économique et militaire d'un pays.\\

Par nature les réseaux routiers sont des structures de grande envergure, il est
donc difficile d'en avoir une vision globale à chaque instant et donc de
pouvoir surveiller leur état.\\
Il existe tout de même de nos jours différentes techniques permettant de
détecter des dégâts sur les chaussées. Les plus simples, basées sur l'analyse
d'images ou de données accélérométriques, permettent de déceller les
dégradations en surface, d'autres plus précises et plus couteuses permettent
aussi d'analyser la structure de la route en profondeur.\\

Le but de ce projet est de concevoir et développer un système informatique qui
permet de détecter automatiquement tout type de dégradation sur les routes afin
qu'elles puissent être signalées et réparées dans les plus brefs délais.\\
Ce système devra présenter un faible coût de déploiement et impact sur
l'environnement tout en restant peu intrusif du point de vue de l'utilisateur
(contraintes d'utilisation et vie privée).