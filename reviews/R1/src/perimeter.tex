\section{Périmètre du Projet}

\subsection{Coût}
Pour le moment, il n'est pas envisagé d'effectuer des coûts spécifiques afin de
réaliser le projet. En effet,
le projet se trouve encore dans la phase de compréhension et de planification.
L'état de l'art nous permet
d'avoir une idée de l'envergure du projet et des outils et méthodes à
disposition pour le réaliser, cependant
il est difficile à ce stade du projet de se projeter sur l'utilisation
particulière d'un outil qui demandera alors
un coût supplémentaire.
En effet, créer une application mobile sur Androïd Studio ne demande pas de
frais supplémentaires, donc
sa réalisation peut se faire gratuitement. Il est possible qu'il y ait frais
liés à l'utilisation de fonctionnalités
ou autre, mais cela se verra au fur et à mesure de la réalisation de
l'application.
De plus, l'école nous fournit le robot téléguidé Scout Mini AgileX, dont le
prix est assez élevé. Il n'y a
donc pour l'instant pas d'utilité à envisager l'achat d'un autre appareil pour
collecter les données et faire
des expériences.
Il est néanmoins possible qu'il y ait besoin d'acheter du carton ou des plaques
pour simuler une forme
particulière de la route, afin de faire des expériences avec le robot Scout.
Cependant cela dépend de la
qualité de détérioration du parking de l'école. Si le parking présente trop peu
d'hétérogénéité, il sera alors
utile d'effectuer des tests de tenue de route avec le robot Scout dans une
forme de route particulière. Dans
ce cas, il faudra pour cela simuler des détériorations de la route à l'aide de
carton ou de plaques
(métalliques, en plastique, ...).
\subsection{Délai}
Le planning avec les users stories, qui permet d'ordonner les tâches, a été
effectué. Cela nous donne une
très bonne idée de la priorité de chaque tâche, et donc de savoir quelle tâche
doit être plus considérée que
les autres, et pour laquelle il faudra consacrer éventuellement plus de temps.
\subsection{Qualité}
Afin que l'objectif du projet soit rempli, il faut que tous les types
de
détérioration soient pris en
compte dans notre étude. En effet, il ne faut pas que le modèle effectue une mauvaise classification.
Cela sera un problème lorsqu'un un conducteur se retrouvera face à cette
détérioration particulière.