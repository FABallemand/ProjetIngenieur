\section{Etat de l'Art}
L'intelligence artificielle est de plus en plus utilisée et devient de plus en plus performante. Les modèles d'IA actuels sont capables de répondre à des tâches complexes ayant de fortes répercutions. Les applications sont diverses : traitement de signaux audios \cite{rtx_voice}, classification d'images médicales \cite{inria_medical} ou détection de défauts sur des chaines de production par traitement de mesures \cite{valeo}.\\

Il existe déjà de nombreuses applications au domaine routier. On retrouve en autre la détection de conduite en état d'ivresse \cite{drunk_1}\cite{drunk_2} ou le contrôle de véhicules \cite{xxii} sans oublier les véhicules autonomes qui ne cessent d'évoluer \cite{nvidia_car}.\\

D'autres applications du machine learning utilisent des données accélérométriques. L'essor des montres connectées a notamment révélé l'utilité de ce type de données pour la détection de chute \cite{fall_1} \cite{fall_2}.\\

L'analyse de l'état des routes et la détection de dégradations a été l'objet de nombreuses recherches. Les données parfois collectées avec plusieurs accéléromètres \cite{road_1} sont généralement utilisées dans le domaine fréquentiel \cite{road_2}\cite{road_3}. L'utilisation des capteurs accélérométriques contenus dans les smartphones s'est déjà montrée comme une alternative pertinente aux capteurs externes \cite{road_4}.\\