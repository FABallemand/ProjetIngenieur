\section{Conclusion}
Il est nécessaire de garder un réseau routier en bon état afin de limiter les gênes sur les utilisateurs, de réduire la pollution et de favoriser la compétitivité d'un pays. Toutefois, l'étendue du réseau routier rend cette tâche complexe.\\

Afin de répondre à ce besoin, une application utilisée par un grand nombre de conducteurs permettant de collecter des données (en particulier des données accélérométriques) représentant l'état des routes sera développée. Cette application sera reliée à un serveur sur lequel une intelligence artificielle analysera et classifiera les données afin de déceler les dégâts sur les routes. Cette même application, utilisée par des ouvriers voiries, permettra de faciliter l'entretien des routes. Une collecte d'informations provenant des ouvriers permettrait d'améliorer les performances de l'intelligence artificielle.\\

Dans un premier temps, un premier modèle d'intelligence artificelle sera entrainé grâce à des données recueillies par un dispositif Arduino monté sur un robot téléguidé. Ce premier modèle permettra de tester les capacités d'une intelligence artificielle à détecter les dégradations sur les chaussées. Une application pour smartphone Android pourra ensuite être créee pour simuler le déploiement d'un système participatif communautaire.\\