\section{Introduction}
Road infrastructures are key when it comes to traveling. Whether it is for daily commuting or one-time journeys, millions of people drive their vehicle on the road in order to go from a point A to a point B.\\
There is no denying that the state of deterioration of roads has a huge impact on the security of the drivers and passengers. An unexpected hole on a road can lead a conductor to change direction abruptly or loose control of the vehicle.\\
The effect of a poorly maintained road on vehicle is usualy overlooked but it seems logical that holes and bumps on a road are likely to cause damage on cars reducing security and increasing maintenance costs on the vehicle.\\
At a larger scale, transportations can be slowed down by deteriorated roads meaning the entiere process of economical exchanges is running at a slower pace, hurting the economy of cities or even countries.\\
Finally, the military force of a country can be evaluated by the state of roads networks. In case of emergency, military forces need to move quickly. Once again, the state of the roads is a key factor.\\

The goal of this project is to develop an AI-based solution in order to facilitate road maintenance. By training an AI to recognize degradation on a road, road-wrokers could more easily service roads and thus improve security and user experience.\\
In this study, the AI will be mostly trained on acceleration data measured on vehicles. In order to work properly, the model must be able to detect various degradations (bumps and obstacles, holes and cracks as well as gravel) regardless of the type of vehicle the data is coming from.\\
For the purpose of the study, two methods will be used in order to collect acceleration data. First, using an Arduino and an \textit{Inertial Measurement Unit} (IMU). In a second time, by using smartphones accelerometers.
% Different approaches can be used in order to collect acceleration data. The easiest is to create a simple device with an IMU that records data localy requiring human interaction in order to transfer data. Another way to access such type of data is to use accelerometers located in smartphones...
\\
As a proof of concept, a smartphone application will be created and will demonstrate the effectiveness of this road quality assessment method.\\