\section{Introduction}
Les infrastructures routières sont essentielles lorsqu'il s'agit de voyager. Qu'il s'agisse de déplacements quotidiens ou de voyages ponctuels, des millions de personnes conduisent leur véhicule sur la route pour aller d'un point A à un point B. Il est indéniable que l'état de détérioration des routes a un impact considérable sur la sécurité des conducteurs et des passagers. Un trou inattendu sur une route peut amener le conducteur à changer brusquement de direction ou à perdre le contrôle de son véhicule. L'effet d'une route mal entretenue sur un véhicule est généralement négligé, mais il semble logique que les trous et les bosses sur une route soient susceptibles d'endommager les véhicules, réduisant ainsi la sécurité et augmentant les coûts d'entretien du véhicule. À plus grande échelle, les transports peuvent être ralentis par des routes détériorées, ce qui signifie que l'ensemble du processus d'échanges économiques se déroule à un rythme plus lent, ce qui nuit à l'économie des villes ou même des pays. Enfin, la force militaire d'un pays peut être évaluée en fonction de l'état des réseaux routiers. En cas d'urgence, les forces militaires doivent se déplacer rapidement. Là encore, l'état des routes est un facteur clé.

L'objectif de ce projet est de développer une solution basée sur l'IA afin de faciliter l'entretien des routes. En formant une IA à reconnaître les dégradations sur une route, les cantonniers pourraient plus facilement entretenir les routes et ainsi améliorer la sécurité et l'expérience des usagers.\\
Dans cette étude, l'IA sera principalement entraînée sur les données d'accélération mesurées sur les véhicules. Pour fonctionner correctement, le modèle doit être capable de détecter diverses dégradations (bosses et obstacles, trous et fissures ainsi que graviers), quel que soit le type de véhicule d'où proviennent les données.\\
Pour les besoins de l'étude, deux méthodes seront utilisées pour collecter les données d'accélération. Tout d'abord, en utilisant une carte Arduino et une \textit{unité de mesure inertielle} (IMU). Dans un second temps, en utilisant l'accéléromètre contenu dans les smartphones.\\
Comme preuve de concept, une application pour smartphone sera créée et démontrera l'efficacité de cette méthode d'évaluation de la qualité des routes.