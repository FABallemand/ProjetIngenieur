\section{Conclusion}

Dans le cadre de ce projet de mesure des infrastructures routières, nous avons exploré différentes approches, y compris l'utilisation de modèles d'apprentissage automatique tels que les CNN et les LSTM. Les résultats obtenus avec notre modèle CNN sont très encourageants, avec une précision de 93\% dans la classification des obstacles. Cependant, des obstacles techniques ont entravé l'entraînement complet du modèle LSTM. Ces problèmes techniques feront l'objet d'une résolution ultérieure afin de permettre une évaluation complète de ce modèle.

En fin de compte, ce projet représente une étape cruciale vers le développement d'un système de détection d'obstacles sur les infrastructures routières, ce qui pourrait avoir un impact significatif sur la sécurité routière et l'entretien des routes. Les découvertes et les modèles développés ici serviront de base pour des travaux futurs visant à améliorer encore la précision et l'efficacité de notre système.