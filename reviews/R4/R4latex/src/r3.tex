\section{Rappels R1, R2 et R3}

Les premières étapes de notre projet ingénierieur ont été consacrées à la compréhension de la problématique, des enjeux et des défis associés. Des recherches approfondies ont été menées pour explorer les méthodes d'évaluation de la qualité des routes, afin de recueillir des informations sur les solutions existantes et les méthodologies pertinentes.

L'organisation du projet s'est ensuite concrétisée grâce à la définition des objectifs des sprints, l'utilisation de "user stories" et l'évaluation de la complexité des tâches. Ces sprints ont servi à planifier l'avancement global du projet. Le projet a été divisé en trois parties clés : la collecte de données, la construction et l'entraînement d'une intelligence artificielle, et le développement d'une application Android.

Nous avons commencé par la collecte de données à l'aide d'un dispositif basé sur Arduino, enregistrant les premières données et les comparant avec d'autres sources telles que les données de M. Helbert et un jeu de données en ligne. Ces enregistrements ont jeté les bases de notre jeu de données initial.

En parallèle, le développement d'une application Android a été entamé pour permettre une collecte de données plus précise et efficace. 

Dans le rapport 3, nous avons noté que grâce à l'expérience acquise lors de la première collecte de données et aux progrès réalisés dans le développement de l'application Android, nous avons pu construire un jeu de données plus robuste et de meilleure qualité. Cependant, pour entraîner efficacement un modèle d'intelligence artificielle, nous avons dû augmenter artificiellement la taille du jeu de données en utilisant différentes techniques.

Une phase importante de pré-traitement des données a été réalisée pour préparer les données à l'entraînement de l'IA, notamment l'automatisation de la création de séquences numériques et l'étiquetage manuel des données en fonction des informations recueillies lors des enregistrements. De plus, une phase ultérieure de pré-traitement a été identifiée pour uniformiser la longueur des séquences.

Pour la suite du projet, plusieurs approches de classification des séquences seront explorées, notamment des méthodes classiques telles que les approches d'apprentissage profond comme les réseaux LSTM, ou CNN ainsi que des techniques de détection d'anomalies. Une évaluation des modèles sera réalisée pour sélectionner la meilleure approche pour résoudre la problématique du projet. Le modèle sélectionné sera déployé, et des efforts d'amélioration continue seront entrepris pour affiner ses performances au fil du temps.
