\section{Réorganisation}

Lors de la planification des différentes étapes du projet, incluant la spécification des "user stories" et la segmentation en sprints pour définir les objectifs intermédiaires entre les rapports successifs, notre équipe initiale était composée de trois membres. Toutefois, en raison de circonstances imprévues liées au parcours académique de Fabien ALLEMAND, qui a décidé de suivre un cheminement différent de ses études en troisième année à Telecom Physique Strasbourg, une réorganisation de notre programme initial s'est avérée nécessaire. Cette réorganisation a été entreprise afin d'adapter notre planification à deux personnes, tout en préservant au mieux la continuité et l'efficacité du projet.

Voici le nouveau diagramme de Gantt correspondant à nos besoins: 
\begin{figure}[H]
    \centering
    \includegraphics[width=0.75\linewidth]{Diagramme de Gantt (Présentation (169)).jpg}
    \caption{Diagramme de Gantt}
    \label{gantt}
\end{figure}

Cette fois, le planning s'étend de septembre à janvier. Dans le but d'optimiser notre efficacité, nous avons structuré notre plan de manière à ce que deux tâches puissent être exécutées en parallèle à tout moment. Cette approche nous permet de répartir les responsabilités au sein de l'équipe, tout en maintenant la possibilité d'apporter un soutien mutuel si nécessaire pour la réalisation des tâches individuelles.

Plus spécifiquement, notre plan actuel prévoit la phase d'élaboration d'un modèle d'intelligence artificielle au cours du mois de septembre, suivie d'une période d'amélioration continue tout au long du mois d'octobre. Parallèlement à cette activité, à partir du mois d'octobre, nous nous engageons à poursuivre l'amélioration de l'application, en la rendant de plus en plus fonctionnelle et performante. Enfin de novembre à janvier nous créerons un dashboard, qui fournira une vue synthétique et visuelle des données permettant de faciliter la prise de décision.
