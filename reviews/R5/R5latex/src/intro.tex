\section{Introduction}
The road infrastructures are crucial when it comes to traveling. Whether it's daily commutes or occasional trips, millions of people drive their vehicles from point A to point B on the road. It's undeniable that the deteriorating state of roads has a considerable impact on the safety of drivers and passengers. An unexpected hole on a road can cause the driver to suddenly change direction or lose control of their vehicle. The effect of a poorly maintained road on a vehicle is generally overlooked, but it seems logical that potholes and bumps on a road are likely to damage vehicles, thereby reducing safety and increasing vehicle maintenance costs. On a larger scale, transportation can be slowed down by deteriorating roads, meaning the entire process of economic exchange operates at a slower pace, adversely affecting the economy of cities or even countries.\\

In this study, the AI will primarily be trained on acceleration data measured on vehicles. For the model to function correctly, it needs to be capable of detecting various deteriorations (bumps and obstacles, holes and cracks, as well as gravel), regardless of the type of vehicle from which the data originates. For the purposes of the study, two methods will be used to collect acceleration data. Firstly, by using an Arduino board and an Inertial Measurement Unit (IMU). Secondly, by utilizing the accelerometer contained within smartphones.\\

As a proof of concept, a smartphone application will be created and will demonstrate the effectiveness of this method for assessing road quality."