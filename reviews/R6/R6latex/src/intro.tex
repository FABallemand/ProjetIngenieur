\section{Introduction}
Les infrastructures routières jouent un rôle essentiel dans la mobilité quotidienne et les déplacements ponctuels. Des millions de conducteurs empruntent les routes pour atteindre leur destination. La détérioration de ces routes a un impact majeur sur la sécurité, avec des conséquences potentielles telles que des changements brusques de direction ou la perte de contrôle du véhicule en raison de trous inattendus. Cette négligence des routes endommagées peut entraîner des dommages aux véhicules, augmentant ainsi les coûts d'entretien. À une échelle plus large, des routes détériorées ralentissent les transports, affectant les échanges économiques et nuisant à l'économie locale et nationale. De plus, l'état des routes peut influencer l'évaluation de la force militaire d'un pays, particulièrement lors d'urgences nécessitant un déploiement rapide des forces.

L'objectif de ce projet est de développer une solution basée sur l'IA afin de faciliter l'entretien des routes. En formant une IA à reconnaître les dégradations sur une route, les cantonniers pourraient plus facilement entretenir les routes et ainsi améliorer la sécurité et l'expérience des usagers.\\
Dans cette étude, l'IA sera principalement entraînée sur les données d'accélération mesurées sur les véhicules. Pour fonctionner correctement, le modèle doit être capable de détecter diverses dégradations (bosses et obstacles, trous et fissures ainsi que graviers), quel que soit le type de véhicule d'où proviennent les données.\\
Pour les besoins de l'étude, deux méthodes seront utilisées pour collecter les données d'accélération : une carte Arduino munie d'une \textit{unité de mesure inertielle} (IMU) et un smartphone en utilisant l'accéléromètre contenu dans son dispositif.\\
Comme preuve de concept, une application pour smartphone sera créée et démontrera l'efficacité de cette méthode d'évaluation de la qualité des routes.