\section{Rappels R1 à R5}

Le projet a débuté par la collecte de données via Arduino et le développement d'une application Android, suivi de l'exploration des méthodes de classification de séquences, notamment l'utilisation d'un CNN. La réorganisation de l'équipe a entraîné des ajustements de calendrier. La revue n°5 a souligné le développement d'une nouvelle application Android intégrée à Google Drive pour simplifier l'utilisation des données.

La revue n°5 a introduit une approche de classification multi-classe avec un CNN utilisant une méthode de fenêtre glissante, montrant des résultats prometteurs pour des séquences plus courtes, mais identifiant des limitations pour les séquences plus longues. Cette dernière revue a également  abordé les méthodes envisagées pour la validation des données augmentées, dans le but de confirmer leur intégrité. 

Globalement, le projet a réalisé des avancées significatives sur le plan du développement d'applications et de l'exploration de la classification des séquences. 
